\begin{appendix}
\section{Proof of bound on writes}
\noindent
\textbf{Lemma 1:}
$\forall a, b$ such that $a>0$, $b>0$ and $a\leq b$\\
$(a+b)\log_2 (a+b) \geq a\log_2 a + b\log_2 b +2a $
\\
\noindent
\textbf{Proof}
\\
$(a+b)\log_2 (a+b)$ ? $a\log_2 a + b\log_2 b +2a $
\\$\equiv a\log_2 (a+b) + b\log_2 (a+b)$ ? $a\log_2 a + b\log_2 b +2a $
\\$\equiv a\log_2 (\frac{a+b}{a}) + b\log_2 (\frac{a+b}{b})$ ? $2a$
\\$\equiv \frac{a}{a+b}\log_2 (\frac{a+b}{a}) + \frac{b}{a+b}\log_2 (\frac{a+b}{b})$ ? $2\frac{a}{a+b}$ {\hfill Divide by $(a+b)$}
\\Define $x=\frac{a}{a+b}$ and $y=\frac{b}{a+b}$
\\Then $x+y=1$, $x>0$, $y>0$, $x\leq y$, $0<x\leq 0.5$ and $y=1-x$
\\$\equiv x\log_2 (\frac{1}{x}) + y\log_2 (\frac{1}{y})$ ? $2x$
\\$\equiv -x\log_2x - y\log_2y$ ? $2x$
\\$\equiv -x\log_2x - (1-x)\log_2(1-x)$ ? $2x$
\\Let $f_1=(-x\log_2x - (1-x)\log_2(1-x))$ and $f_2=2x$. It can be shown formally that for $0<x\leq 0.5$ $f_1 \geq f_2$. The graph of $f_1$ and $f_2$ are shown in
Figure ~\ref{fig:proof_graphs} .

\begin{figure}[htbp]
\includegraphics[height = 4cm]{proof_graphs.png}\centering
\caption{Comparison of functions}
\label{fig:proof_graphs}
\end{figure}

\noindent Hence
\\$\equiv -x\log_2x - (1-x)\log_2(1-x)$ $\geq$ $2x$
\\$\equiv (a+b)\log_2 (a+b) \geq a\log_2 a + b\log_2 b +2a $
\\


%\vspace{5mm}
\noindent
\textbf{Corollary 1:}
In Lemma 1, taking $(a+b)$ as $n$ and $s$ as $a$, we have\\
$s \log_2{\frac{s}{m}} + (n-s) \log_2{\frac{n-s}{m}} +2s \leq n \log_2{\frac{n}{m}}$\\
where  $1 \leq s \leq \frac{n}{2}, m \geq 2, n \geq 2$
\\Terms containing $m$ will cancel out from both side.
\\

%\vspace{5mm}
\noindent
\textbf{Lemma 2:}
$\forall k, m$ such that $k \geq 1$, $m \geq 2$ and $k\leq m$:\\
$ k < (m+k) \log_2{\frac{m+k}{m}} $
\\
\noindent
\textbf{Proof}
\\
$k$ ? $(m+k)\log_2{\frac{m+k}{m}}$
\\ We can write $k=x\times m$ where $0<x\leq 1$
\\$\equiv x\times m$ ? $(m+x\times m)\log_2{\frac{m+x\times m}{m}}$
\\$\equiv x$ ? $(1+x)\log_2{(1+x)}$
\\Let $f_1=(1+x)\log_2{(1+x)}$ and $f_2=x$. It can be shown formally that for $0<x\leq 1$, $f_2 < f_1$. The graphs of $f_1$ and $f_2$ are shown in
Figure ~\ref{fig:proof_graphs} .

\noindent Hence
\\$\equiv x < (1+x)\log_2{(1+x)}$
\\$\equiv k < (m+k) \log_2{\frac{m+k}{m}} $
\\

%\vspace{5mm}
%\begin{array}{l l l}
  
\noindent
\newtheorem{theorem}{Theorem}
\begin{theorem}
  If $W(n) = \left\{
  \begin{array}{p{4cm}p{2cm}p{2cm}}
    0 & \text{if $n \leq 1$}\\
    n &  \text{if $1 < n \leq m $}\\
    $\max_{\substack{1 \leq s \leq \lfloor \frac{n}{2}\rfloor}}
     (W(s)+ W(n-s)+2s)$
      &  \text{if $n>m$}\\
  \end{array}
  \right.$
  where $m\geq 2$
  \\ Then $W(n)< (i+2)n$
  \\ where $i=max\left(0, \log_2{\frac{n}{m}}\right)$

\end{theorem}
\noindent
\begin{proof}
We first prove the theorem for smaller value of n on a case by case basis. For larger value of $n$, we will apply Corollary 1 directly.

\noindent
\textbf{Case 1: $1\leq n \leq m$}\\
$W(1)=0$ which is less than $(i+2)n=2$ \\
$W(n)=n$ which is less than $(i+2)n=2n$ 
\\
\noindent
\textbf{Case 2: $m < n \leq 2m$}\\
In this case $W(n)=\max_{\substack{1 \leq s \leq \lfloor \frac{n}{2}\rfloor}}(W(s)+ W(n-s)+2s)$ . 
Let's assume that $n=m+k$ where $1\leq k \leq m$. Now we have the following possibilities:
\\
\noindent
\textbf{Case 2A: $ s \geq k$}\\
Hence $W(n)=\max_{\substack{k \leq s \leq \lfloor \frac{m+k}{2}\rfloor}}(W(s)+ W(n-s)+2s)$
\\
$ = \max_{\substack{k \leq s \leq \lfloor \frac{m+k}{2}\rfloor}}(s+ (n-s)+2s)$
\\
$ = \max_{\substack{k \leq s \leq \lfloor \frac{m+k}{2}\rfloor}}(n+2s)$
\\
$ = (n+2\frac{m+k}{2})$
$ = 2n$
$< (i+2)n$

%\\
\noindent
\textbf{Case 2B: $ s < k$}\\
Hence $W(n)=\max_{\substack{1 \leq s \leq k-1}}(W(s)+ W(n-s)+2s)$
\\
$ = \max_{\substack{1 \leq s \leq k-1}}(s+ W(m+k-s)+2s)$
\\
$W(m+1)=2(m+1)= 0+2(m+1)$
\\
$W(m+2)=max\left\{ 
\begin{array}{l}
2(m+2)\\
2*1+W(1)+W(m+1)
\end{array} \right.$

$=max\left\{ 
\begin{array}{l}
2(m+2)\\
2*1+0+0+2(m+1)
\end{array}\right.$
$=0+2(m+2)$
\\
$W(m+3)=max\left\{ 
\begin{array}{l}
2(m+3)\\
2*1+W(1)+W(m+2)\\
2*2+W(2)+W(m+1)
\end{array}\right.$

$=max\left\{ 
\begin{array}{l}
2(m+3)\\
2*1+0+0+2(m+2)\\
2*2+2+0+2(m+1)
\end{array} \right.$
$=2+2(m+3)$
\\
$W(m+4)=max\left\{ 
\begin{array}{l}
2(m+4)\\
2*1+W(1)+W(m+3)\\
2*2+W(2)+W(m+2)\\
2*3+W(3)+W(m+1)\\
\end{array}\right.$

$=max\left\{ 
\begin{array}{l}
2(m+4)\\
2*1+0+2+2(m+3)\\
2*2+2+0+2(m+2)\\
2*3+3+0+2(m+1)
\end{array} \right.$
$=3+2(m+3)$
\\
$W(m+k)=g(k)+2(m+k)$ where $g(1)=0$, $g(2)=0$, $g(3)=2$, $g(4)=3$
\\
For $k>4$\\
$W(m+k)=max\left\{ 
\begin{array}{l}
2(m+k)\\
2*s+W(s)+W(m+k-s)\\  1\leq s \leq k-1 
\end{array} \right.$

$=max\left\{ 
\begin{array}{l}
2(m+k)\\
2*1+0+g(k-1)+2(m+k-1)\\
2*s+s+g(k-s)+2(m+k-s)\\  2\leq s \leq k-1
\end{array}\right.$
$=g(k)+2(m+k)$
\\
where $g(k)=max(g(k-1),max(s+g(k-s))$ for $2\leq s\leq k-1)$
\\Hence $g(k)=k-1$ for $k>2$ and $g(1)=g(2)=0$
\\Hence $W(m+k)\leq (k-1)+2(m+k)$ $1\leq k \leq m$  
\\Now by using Lemma 2
\\$W(m+k) < ((m+k)\log_2{\frac{m+k}{m}}+2(m+k))$
\\$\equiv W(n) < (i+2)n $

%\\
\noindent
\textbf{Case 3: $2m < n \leq 3m$}\\
In this case $W(n)=\max_{\substack{1 \leq s \leq \lfloor \frac{n}{2}\rfloor}}(W(s)+ W(n-s)+2s)$. Now we will use mathematical induction. Let's assume that $n=2m+k$ where $1\leq k \leq m$ .
\\
\noindent
\textbf{Case 3A: $ m < s \leq \frac{m+k}{2} \implies s>k$}\\
By induction, $W(s)<(i_s+2)s$ and $W(n-s)<(i_{n-s}+2)(n-s)$. Since both $s$ and $(n-s)$ are greater than $m$,  $i_s$ and $i_{n-s}$ would be defined by logarithmic term. Hence
\\$(2s+W(s)+W(n-s))<(2s+ s\log_2{\frac{s}{m}}+2s+(n-s)\log_2{\frac{n-s}{m}} +2(n-s))$
\\$=2n+ (s\log_2{\frac{s}{m}}+(n-s)\log_2{\frac{n-s}{m}} +2s)$
\\$<2n + n\log_2{\frac{n}{m}}$ {\hfill By Corollary 1}
\\$=(i_n + 2)n$

%\\
Hence $W(n)<(i_n+2)n$

\noindent
\textbf{Case 3B: $ k < s \leq m $}\\
In this case $m < (n-s)\leq 2m$, Hence
\\$W(n)=\max_{\substack{k+1 \leq s \leq m}}(W(s)+ W(n-s)+2s)$
\\$=\max_{\substack{k+1 \leq s \leq m}}(s + 2(n-s) + (n-s-m-1) + 2s)$ {\hfill From Case 2}
\\$=\max_{\substack{k+1 \leq s \leq m}}(2n + (n-m-1))$
\\$=2n + (n-m-1)$
\\$<2n + n $ {\hfill $\because n>m$}
\\$<2n + n\log_2{\frac{n}{m}} $ {\hfill $\because \log_2{\frac{n}{m}}>1$}
\\$=(i_n + 2)n$
\\Hence $W(n)<(i_n+2)n$

%\\
\noindent
\textbf{Case 3C: $ s \leq k \implies s<m$}\\
In this case $(n-s)\geq 2m$ and $W(n)=\max_{\substack{1 \leq s \leq  k}}(s + W(n-s)+2s)$
\\
By induction, $W(n-s)<(\log_2{\frac{n-s}{m}} +2)(n-s)$ since $(n-s)>m$ . Hence, the $i_{n-s}$ would be defined by a logarithmic value. Hence
\\$(s + W(n-s)+2s)<(s+2s+(\log_2{\frac{n-s}{m}} +2)(n-s))$
\\$=s+2s+2n-2s+(n-s)\log_2{\frac{n-s}{m}}$
\\$=2n + n\log_2{\frac{n-s}{m}} +s -s\log_2{\frac{n-s}{m}}$
\\$=2n + n\log_2{\frac{n-s}{m}} +s -s\log_2{\frac{n-s}{m}}$
\\$=2n + n\log_2{\frac{n-s}{m}} +s -s(value \geq 1)$
\\$\leq 2n + n\log_2{\frac{n-s}{m}}$
\\$< 2n + n\log_2{\frac{n}{m}}$ {\hfill $\because \log_2{\frac{n}{m}} > \log_2{\frac{n-s}{m}}$}
\\$=(i_n + 2)n$
\\Hence $W(n)<(i_n+2)n$


%\\
\noindent
\textbf{Case 4: $ n > 3m$}\\
In this case $W(n)=\max_{\substack{1 \leq s \leq \lfloor \frac{n}{2}\rfloor}}(W(s)+ W(n-s)+2s)$.
\\
\noindent
\textbf{Case 4A: $ s \leq m $ }\\
It is same as Case 3C because in this case $(n-s)\geq 2m$.

\noindent
\textbf{Case 4B: $ s > m $}\\
This is the same as Case 3A because in this case, both $s$ and $(n-s)$ are greater than $m$. Hence, $i_s$ and $i_{n-s}$ would be defined by a logarithmic term. So Corollary 1 can be applied directly.
\\Hence $W(n)<(i_n+2)n$


\vspace{5mm}
\noindent

\end{proof}
\end{appendix}